\documentclass{article}
\usepackage{amsmath, amssymb}
\usepackage[retainorgcmds]{IEEEtrantools}
\usepackage{filecontents}
\usepackage{hyperref}
\author{Derek Kuo, Henry Milner}
\title{CS267 HW2: Particle Simulation}
\date{3/13/15}

% Some functions for general use.

\def\seqn#1\eeqn{\begin{align}#1\end{align}}

\newcommand{\vecName}[1]%
  {\boldsymbol{#1}}

\newcommand{\io}%
  {\text{ i.o. }}

\newcommand{\eventually}%
  {\text{ eventually }}

\newcommand{\tr}%
  {\text{tr}}

\newcommand{\Cov}%
  {\text{Cov}}

\newcommand{\adj}%
  {\text{adj}}

\newcommand{\funcName}[1]%
  {\text{#1}}

\newcommand{\hasDist}%
  {\sim}

\DeclareMathOperator*{\E}%
  {\mathbb{E}}

\newcommand{\Var}%
  {\text{Var}}

\newcommand{\std}%
  {\text{std}}

\newcommand{\grad}%
  {\nabla}

\DeclareMathOperator*{\argmin}{arg\,min}

\DeclareMathOperator*{\argmax}{arg\,max}

\newcommand{\inprod}[2]%
  {\langle #1, #2 \rangle}

\newcommand{\dd}[1]%
  {\frac{\delta}{\delta#1}}

\newcommand{\Reals}%
  {\mathbb{R}}

\newcommand{\indep}%
  {\protect\mathpalette{\protect\independenT}{\perp}} \def\independenT#1#2{\mathrel{\rlap{$#1#2$}\mkern2mu{#1#2}}}

\newcommand{\defeq}%
  {\buildrel\triangle\over =}

\newcommand{\defn}[1]%
  {\emph{Definition: #1}\\}

\newcommand{\example}[1]%
  {\emph{Example: #1}\\}

\newcommand{\figref}[1]%
  {\figurename~\ref{#1}}

\newtheorem{theorem}{Theorem}[section]
\newtheorem{lemma}[theorem]{Lemma}
\newenvironment{proof}[1][Proof]{\begin{trivlist}
\item[\hskip \labelsep {\bfseries #1}]}{\end{trivlist}}

\begin{filecontents}{\jobname.bib}
%@book{foo
%}
\end{filecontents}

\begin{document}
\maketitle

\section{Introduction}

\section{Serial Implementation}
% Describe algorithm: Grid data structure
% Plot: O(n) scaling: n = 250, 1000, 4000, 16000, 64000, 256000
% Plot: Compare with serial-naive: n = 100, 200, 400, 800, 1600, 3200, 6400, 12800
% Plot: Hopper versus laptop: n = 250, 1000, 4000, 16000, 64000, 256000

\section{OpenMP Implementation}
% Describe algorithm: Grid data structure; threads handle subgrids; communication between threads is implicit and of size O(particles that moved across subgrid boundaries)
% Describe synchronization: Locking on grid inserts (monotonic operation) plus barriers before/after reads.
% Plot: O(n) scaling on Hopper: n = 250, 1000, 4000, 16000, 64000, 256000
% Plot: Weak scaling(?) on Hopper: p = 1, 2, 4, 8, 16, 24; n = 250, 1000, 4000, 16000, 64000 (* number of cores) [use more / less if possible on Hopper]
% Plot: Strong scaling(?) on Hopper: p = 1, 2, 4, 8, 16, 24; n = 250, 1000, 4000, 16000, 64000, 256000

\section{MPI Implementation}
% Plot: O(n) scaling on Hopper: n = 1000, 4000, 16000, 64000, 256000
% Plot: Weak scaling(?) on Hopper: p = 1, 4, 16, 64, 256
% Plot: Strong scaling(?) on Hopper: p = 1, 4, 16, 64, 256
% Plot: Compare grid- and non-grid O((n/p)^2) code for small problem sizes.

\section{CUDA Implementation}
% Describe algorithm.
% Describe sorting algorithm and plans for future work.
% Plot: Compare with naive: n = 250, 1000, 4000, 16000, 64000(?)
% Plot: O(n) scaling on Stampede: n = 250, 1000, 4000, 16000, 64000, 256000, 1024000, 4096000
% Plot: Laptop versus Stampede: n = 250, 1000, 4000, 16000, 64000, 256000, 1024000, 4096000 [last two on Stampede only due to memory constraints]

\section{Observations}

\subsection{Performance Comparison}
% One plot with all algorithms: serial-naive, serial, OpenMP (p=24), MPI (p=24), MPI (p=256), CUDA-naive, CUDA, CUDA-laptop: n = 250, 1000, 4000, 16000, 64000, 256000, 1024000, 4096000 [as applicable]

\subsection{Programming Environment}
% OpenMP: One difficulty was that there are no fine-grained locks without full memory barriers.
% CUDA: We used Thrust.  Provides a mapreduce-style interface, though not really mapreduce (since things are necessarily mutable).  Some opaque crashes.  Mapreduce is not quite the right abstraction for us; something like GraphLab would have been better.  As is, we had to use a general sort, which technically introduces O(n log n) scaling, though this was unnoticeable in our tests.  And we had to break out of Thrust for the force simulation.

%\bibliographystyle{plain}
%\bibliography{\jobname}

\end{document}